\chapter{Advanced Data Structures}

\section{RMQ versus LCA}

ในข้อนี้ให้ $N, Q$ เป็นจำนวนเต็มบวก

พิจารณาปัญหา RMQ ซึ่งกล่าวไว้ว่า มีอาเรย์ของจำนวนเต็ม $N$ ตัวและมีคำถาม $Q$ คำถาม แต่ละคำถาม ถามว่าระหว่างตัวที่ $l$ ถึง $r$ ตัวที่มีค่าน้อยที่สุดเป็นตัวที่เท่าใด

พิจารณาปัญหา LCA ซึ่งกล่าวไว้ว่า ให้ต้นไม้ระบุราก ประกอบด้วยจุดยอด $N$ จุดยอดและมีคำถาม $Q$ คำถาม แต่ละคำถาม ถามว่าจุดยอดที่เป็นบรรพบุรุษร่วมที่อยู่ตำแหน่งน้อยที่สุด (lowest common ancestor) ของ $u$ กับ $v$ คือจุดยอดใด

\begin{exercise}
อธิบายการทำงานของอัลกอริทึมที่ตอบปัญหา RMQ ได้ โดยเตรียมการ $O(N)$ และตอบคำถามแต่ละคำถามใน $O(\log N)$
\end{exercise}

\begin{exercise}
อธิบายการทำงานของอัลกอริทึมที่ตอบปัญหา LCA ได้ โดยเตรียมการ $O(N \log N)$ และตอบคำถามแต่ละคำถามใน $O(\log N)$
\end{exercise}

พิจารณาการแก้ปัญหา LCA ด้วยวิธีการดังต่อไปนี้

\begin{lstlisting}
std::vector<int> tour;
std::vector<int> children[N+1];
int start[N+1];
void dfs(int current, int parent, int depth){
    start[current] = tour.size();
    tour.push_back(depth);
    for(int child : children[current]){
        if(child == parent) continue;
        dfs(child, current, depth+1);
        tour.push_back(depth);
    }
}
\end{lstlisting}

\begin{exercise}
สังเกตว่า LCA ระหว่างจุดยอด \texttt{u} กับ \texttt{v} คือ RMQ ใน \texttt{tour} ระหว่างตัวที่ \texttt{start[u]} กับ \texttt{start[v]} จงพิสูจน์ว่า \texttt{tour} มีขนาดอยู่ใน $O(N)$ และพิสูจน์ว่าหากปัญหา RMQ มีวิธีแก้โดยการเตรียมการ $T_1(N)$ และตอบคำถามใน $T_2(N)$ แล้ว LCA จะสามารถแก้ได้โดยการเตรียมการใน $T_1(N) + O(N)$ และตอบคำถามใน $T_2(N) + O(1)$
\end{exercise}

ต่อมา พิจารณาโครงสร้างข้อมูล Cartesian Tree ซึ่งเป็นต้นไม้ทวิภาค ที่ทำบนอาเรย์ $A$ นิยามโดย

\begin{enumerate}
    \item สำหรับแต่ละจุดยอดใน Cartesian Tree นี้ จุดยอดที่เป็นลูกของจุดยอดปัจจุบัน จะมีค่ามากกว่าจุดยอดปัจจุบัน หรืออาจกล่าวได้ว่า parent ของแต่ละจุดยอดมีค่าน้อยกว่าจุดยอดนั้น ๆ
    \item หากทำการเดินทางบนต้นไม้แบบตามลำดับ (in-order traversal) แล้วจะได้ผลลัพธ์เป็นอาเรย์ $A$
\end{enumerate}

\begin{exercise}
จงแสดงว่าสำหรับอาเรย์ $A$ ที่ไม่มีสมาชิกซำ้กันเลยจะมี Cartesian Tree สำหรับ $A$ อยู่แบบเดียวเท่านั้น
\end{exercise}

% ต่อมา พิจารณาการแก้ปัญหา RMQ ด้วยวิธีการเตรียมการดังต่อไปนี้

% \begin{enumerate}
%     \item สำหรับ RMQ บนอาเรย์ $A$ สมมติ $A$ มีสมาชิกสองตัวที่มีค่าเท่ากัน กำหนดให้เรียงตามลำดับดัชนี (index) ว่าตัวใดมาก่อน เรียกอาเรย์ใหม่ว่า $A'$
%     \item สร้าง Cartesian Tree สำหรับ $A'$
% \end{enumerate}

% แล้วสำหรับคำถามแต่ละคำถามที่ถามหา RMQ บนช่วง $[l, r]$ ก็ให้ตอบคำถาม LCA บนจุดยอดที่แทนจำนวน $l$ และ $r$

\begin{exercise}
จงพิสูจน์ว่าหากปัญหา LCA มีวิธีแก้โดยการเตรียมการ $T_1(N)$ และตอบคำถามใน $T_2(N)$ แล้ว RMQ จะสามารถแก้ได้โดยการเตรียมการใน $T_1(N) + O(N)$ และตอบคำถามใน $T_2(N) + O(1)$
\end{exercise}

\section{Sparse Table}

\begin{exercise}
สำหรับข้อมูลอาเรย์ $[1, 8, 4, 2, 7, 5, 9, 6]$ จงวาด Sparse Table ที่สอดคล้องกับข้อมูลนี้ (สำหรับการหาค่าน้อยสุดในช่วง)
\end{exercise}

\begin{exercise}
หากต้องการหาค่าน้อยสุดระหว่างตัวที่ $3$ ถึงตัวที่ $5$ (เริ่มนับจาก $1$) จะต้องพิจารณา Sparse Table ช่องไหนบ้าง และได้คำตอบเป็นเท่าใด
\end{exercise}