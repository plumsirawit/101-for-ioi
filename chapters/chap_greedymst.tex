\chapter{Greedy \& MST}

\newcommand{\bene}{\mathsf{B}}
\newcommand{\mb}{\mathsf{MB}}
\section{Consumer Problem}

ในการศึกษาเศรษฐศาสตร์ ซึ่งว่าด้วยการจัดสรรทรัพยากรที่มีอยู่อย่างจำกัดให้กับความต้องการที่มีอยู่อย่างไม่จำกัด ปัญหาหนึ่งที่เป็นปัญหาพื้นฐานและสำคัญ คือปัญหาของผู้บริโภค

สมมติผู้บริโภคมีเงินอยู่จำกัด (เรียกว่า $B$ หน่วย) และต้องการซื้อสินค้าสองชนิด เรียกว่าชนิด \textbf{ก} กับชนิด \textbf{ข} โดยที่ผู้บริโภคจะนิยามฟังก์ชันผลประโยชน์ ตามตารางดังนี้

\begin{center}
\begin{tabular}{|c|c|c|}
\hline
จำนวน (ชิ้น) & ผลประโยชน์ที่ได้จากการซื้อ \textbf{ก} (หน่วย) & ผลประโยชน์ที่ได้จากการซื้อ \textbf{ข} (หน่วย) \\
\hline
1 & 100 & 160 \\
2 & 185 & 310 \\
3 & 260 & 410 \\
4 & 325 & 490 \\
5 & 385 & 520 \\
6 & 435 & 530 \\
7 & 480 & 533 \\
8 & 520 & 535 \\
\hline
\end{tabular}
\end{center}

\begin{exercise}
สมมติว่า สินค้าชนิด \textbf{ก} ราคา 25 บาท และสินค้าชนิด \textbf{ข} ราคา 50 บาท หากผู้บริโภคมีงบ 300 บาท จะมีโอกาสซื้อผลลัพธ์ออกมาได้กี่แบบ
\end{exercise}

\begin{exercise}
สมมติว่า สินค้าชนิด \textbf{ก} ราคา 25 บาท และสินค้าชนิด \textbf{ข} ราคา 50 บาท หากผู้บริโภคมีงบ 300 บาท จะซื้อชนิด \textbf{ก} กี่ชิ้น และชนิด \textbf{ข} กี่ชิ้น เพื่อให้ได้ผลประโยชน์รวมสูงสุด
\end{exercise}

พิจารณาอัลกอริทึมในการตัดสินใจซื้อสินค้า ดังต่อไปนี้
\begin{enumerate}[nosep]
    \item คำนวณผลประโยชน์ส่วนเพิ่ม (marginal benefit) โดยสำหรับสินค้าแต่ละชนิดนั้นผลประโยชน์ส่วนเพิ่มจากการซื้อสินค้าไป $k$ ชิ้น คือผลประโยชน์ที่ได้รับมาเพิ่มขึ้นจากที่มีอยู่ $k-1$ ชิ้น (พิจารณาตารางหลังจากนี้)
    \item คำนวณผลประโยชน์ส่วนเพิ่มต่อเงินที่จะต้องเสียไป (หน่วยต่อบาท) โดยสามารถพิจารณาได้ในตารางหลังจากนี้
    \item จากที่มีเงินอยู่ $B$ บาท และยังไม่ได้ซื้อสินค้าอะไรเลย ค่อย ๆ เลือกซื้อสินค้าที่ \textbf{ผลประโยชน์ส่วนเพิ่มต่อเงินที่ต้องเสียไป} มีค่ามากกว่า
    \item ทำซำ้ จนกว่าเงินจะหมดงบ
\end{enumerate}

\begin{center}
\begin{tabular}{|c|c|c|c|c|c|c|}
\hline
$n$ & $\bene_\text{ก}(n)$ & $\mb_\text{ก}(n)$ & $\mb_\text{ก}(n)/p_\text{ก}$ & $\bene_\text{ข}(n)$ & $\mb_\text{ข}(n)$ & $\mb_\text{ข}(n)/p_\text{ข}$ \\
\hline
1 & 100 & 100 & 4 & 160 & 160 & 3.2 \\
2 & 185 & 85 & 3.4 & 310 & 150 & 3 \\
3 & 260 & 75 & 3 & 410 & 100 & 2 \\
4 & 325 & 65 & 2.6 & 490 & 80 & 1.6 \\
5 & 385 & 60 & 2.4 & 520 & 30 & 0.6 \\
6 & 435 & 50 & 2 & 530 & 10 & 0.2 \\
7 & 480 & 45 & 1.8 & 533 & 3 & 0.06 \\
8 & 520 & 40 & 1.6 & 535 & 2 & 0.04 \\
\hline
\end{tabular}
\end{center}

เมื่อ
\begin{itemize}[nosep]
    \item $n$ แทนจำนวน
    \item $\bene_x(n)$ แทนผลประโยชน์จากการซื้อสินค้าชนิด $x$ ไป $n$ ชิ้น
    \item $\mb_x(n)$ แทนผลประโยชน์ส่วนเพิ่มจากการซื้อสินค้าชนิด $x$ จาก $n-1$ เป็น $n$ ชิ้น
    \item $\frac{\mb_x(n)}{p_x}$ แทนผลประโยชน์ส่วนเพิ่มต่อเงินที่ต้องเสียไป (ในที่นี้คือราคาสินค้า) 
\end{itemize}

จากกรณีข้างต้น อัลกอริทึมดังกล่าวจะซื้อสินค้าชนิด \textbf{ก} จำนวน 6 ชิ้น และชนิด \textbf{ข} จำนวน 3 ชิ้น

\begin{exercise}
พิจารณากรณีทั่วไป หากรับประกันว่าสำหรับทุกจำนวนเต็มบวก $n$, $0 \leq \mb_x(n+1) \leq \mb_x(n)$ เมื่อ $x \in \{\text{ก}, \text{ข}\}$ แล้วจงพิสูจน์ว่าอัลกอริทึมดังกล่าวทำงานถูกต้อง กล่าวคือผลลัพธ์ของอัลกอริทึมดังกล่าวจะให้ผลรวมของผลประโยชน์ของการซื้อสินค้าทั้งสองชนิดออกมาสูงที่สุดเท่าที่จะเป็นไปได้
\end{exercise}

\begin{exercise}
อัลกอริทึมนี้จะยังถูกหรือไม่หากขยายให้มีสินค้ามากกว่าสองชนิด
\end{exercise}

\begin{exercise}
อัลกอริทึมข้างต้นใช้เวลา $O(N)$ ในการคำนวณ เมื่อ $N$ แทนจำนวนสินค้ามากที่สุดที่ซื้อได้ จงออกแบบอัลกอริทึมที่ทำงานได้เหมือนกันแต่ใช้เวลาเพียง $O(\log N)$
\end{exercise}

\begin{bonus}
หาก $\bene_x$ ไม่ใช่แค่ฟังก์ชันที่รับจำนวนเต็มเท่านั้นแต่ขยายเป็นจำนวนจริง และรับประกันว่า $\bene_x$ เป็นฟังก์ชันต่อเนื่องและหาอนุพันธ์ได้สองครั้ง และรับประกันว่า $\bene_x''(t) \geq 0$ สำหรับทุกจำนวนจริง $t > 0$ จงพิสูจน์ว่าเราสามารถใช้อัลกอริทึมข้างต้นแก้ปัญหานี้ (โดยการประมาณค่า) ได้
\end{bonus}

\section{Minimax Path}

ให้กราฟไม่ระบุทิศทางถ่วงนำ้หนักมากราฟหนึ่ง ประกอบด้วย $N$ จุดยอดและ $M$ เส้นเชื่อม ประกอบกับคำถาม $Q$ คำถาม แต่ละคำถามถามว่า หากต้องการเดินทางจากจุดยอด $u$ ไปยังจุดยอด $v$ โดยให้ค่ามากสุดของเส้นเชื่อมแต่ละเส้นที่ใช้เดินทางนั้นออกมามีค่าน้อยที่สุดเท่าที่เป็นไปได้

\begin{exercise}
หาก $Q = 1$ จงออกแบบอัลกอริทึมที่แก้ปัญหานี้ได้ใน $O(M \log N)$
\end{exercise}

\begin{exercise}
จงพิสูจน์ว่าสำหรับทุกคำถาม หากพิจารณาเฉพาะ path จาก $u$ ไปหา $v$ บน MST เท่านั้นจะยังคงให้คำตอบที่ถูกต้องอยู่
\end{exercise}

\begin{exercise}
สำหรับกราฟประเภทต้นไม้ (คนละกราฟกับกราฟหลักของข้อนี้) ที่ประกอบด้วย $N'$ จุดยอด จงออกแบบอัลกอริทึมที่สามารถหาเส้นเชื่อมที่มีค่ามากสุดระหว่างจุดยอด $u$ กับ $v$ ใด ๆ ในกราฟนี้ ภายในเวลา $O(\log N')$ ต่อคำถาม และให้เวลาเตรียมการภายในเวลา $O(N' \log N')$
\end{exercise}

\begin{exercise}
จงออกแบบอัลกอริทึมที่แก้ปัญหานี้ได้ใน $O(M \log N + Q \log N)$
\end{exercise}