\chapter{Mathematical Proof and Induction}

\section{Infinite Number of Primes in an Arithmetic Progression}

\begin{exercise}
จงพิสูจน์ว่ามีจำนวนเฉพาะอยู่เป็นอนันต์ตัว
\end{exercise}

\begin{exercise}
จงพิสูจน์ว่ามีจำนวนเฉพาะอยู่เป็นอนันต์ตัวที่อยู่ในรูป $4k+3$ เมื่อ $k$ เป็นจำนวนเต็ม
\end{exercise}

\section{Well-Ordering Principle and Principle of Mathematical Induction}

สำหรับส่วนนี้ ให้ $\N$ แทนเซตของจำนวนเต็มไม่ลบ คือ $\{0, 1, 2, \dots\}$

\begin{exercise}
ตามสัจพจน์พื้นฐานของจำนวนนับ เราสามารถใช้การอุปนัยเชิงคณิตศาสตร์ได้ จงพิสูจน์หลักการจัดอันดับดีแล้ว ซึ่งกล่าวไว้ว่า สำหรับเซต $S \subseteq \N$ ใด ๆ ที่ $S \ne \emptyset$ จะได้ว่า $S$ มีสมาชิกน้อยสุด
\end{exercise}

\begin{exercise}
สมมติว่าเราไม่เชื่อหลักการอุปนัยเชิงคณิตศาสตร์ แต่มั่นใจว่าหลักการจัดอันดับดีเป็นจริง จงพิสูจน์หลักการอุปนัยเชิงคณิตศาสตร์ กล่าวคือ จงพิสูจน์ว่าสำหรับเซต $K \subseteq \N$ ใด ๆ หาก $0 \in K$ และสำหรับทุกจำนวนเต็มไม่ลบ ถ้า $n \in K$ แล้ว $n+1 \in K$ เราจะสามารถสรุปได้ว่า $\N = K$
\end{exercise}

\section{Sheep Paradox}

\begin{exercise}
สมมติว่าแกะทุกตัวมีสีอยู่ประจำตัวสีเดียว กล่าวคือไม่มีแกะตัวใดที่มีสีสองสีในตัวเดียวกัน พิจารณาการให้เหตุผลดังต่อไปนี้ เพื่อพิสูจน์ว่า "แกะทุกตัวบนโลกมีสีเดียวกัน" ด้วยการอุปนัยเชิงคณิตศาสตร์

\begin{proof}
ขั้นฐาน สมมติว่ามีแกะอยู่ตัวเดียว แกะตัวนั้นย่อมมีสีเดียว ทำให้เซตของแกะทุกเซตขนาด $1$ ประกอบด้วยแกะสีเดียวกัน

ขั้นอุปนัย สมมติให้ $n$ เป็นจำนวนเต็มบวก และสมมติว่าเซตของแกะทุกเซตขนาด $n$ ประกอบด้วยแกะสีเดียวกันทั้งนั้น เราจะแสดงว่าเซตของแกะทุกเซตขนาด $n+1$ ก็ประกอบด้วยแกะสีเดียวกันทั้งนั้น โดยอาศัยว่าพิจารณาเซตของแกะ $n+1$ ตัว เราสามารถหยิบแกะออกไปตัวหนึ่ง จะได้ว่าแกะที่เหลือมีสีเดียวกันเพราะเป็นเซตขนาด $n$ เมื่อใส่แกะตัวเดิมกลับเข้าไปแล้วเอาแกะตัวอื่นออก ก็จะได้ว่าแกะที่เหลือมีสีเดียวกันเพราะเป็นเซตขนาด $n$ จึงสรุปได้ว่าแกะทั้ง $n+1$ ตัวมีสีเดียวกัน

จากหลักการอุปนัยเชิงคณิตศาสตร์ จึงสรุปได้ว่าสำหรับจำนวนเต็มบวก $n$ ใด ๆ แกะทั้ง $n$ ตัวมีสีเดียวกันเสมอ ให้ $N$ เป็นจำนวนแกะบนโลกใบนี้ เนื่องจาก $N$ เป็นจำนวนเต็มบวก จึงได้ว่าแกะทุกตัวบนโลกนี้มีสีเดียวกัน
\end{proof}

เกิดอะไรขึ้นกับบทพิสูจน์ข้างต้น บทพิสูจน์ข้างต้นถูกแล้วหรือ? หากผิด ผิดอย่างไร?
\end{exercise}

\section{A Fixed Point}

กำหนดให้ $f \colon [0, 1] \to [0, 1]$ เป็นฟังก์ชันซึ่ง สำหรับทุก $x, y \in [0, 1]$ หาก $x \leq y$ แล้ว $f(x) \leq f(y)$

นิยาม
\[
A := \{ x \in [0, 1] \colon x \leq f(x) \}
\]

เราจะกล่าวว่า สำหรับเซต $S \subseteq \R$ ใด ๆ $M$ จะเป็นขอบเขตบนของ $S$ ก็ต่อเมื่อสำหรับทุก $x \in S$, $x \leq M$

และเราจะกล่าวว่า สำหรับเซต $S \subseteq \R$ ใด ๆ $m$ จะเป็นขอบเขตบนน้อยสุดของ $S$ ก็ต่อเมื่อ $m$ เป็นขอบเขตบนของ $S$ และไม่มีขอบเขตบนอื่นใดของ $S$ ที่มีค่าน้อยกว่า $m$

\begin{exercise}
จงแสดงว่า $A$ มีขอบเขตบนน้อยสุดใน $[0, 1]$ (ซึ่งจะเขียนแทนด้วยสัญลักษณ์ $a$)
\end{exercise}

\begin{exercise}
จงแสดงว่า $f(a) \leq a$
\end{exercise}

\begin{exercise}
สำหรับ $\varepsilon > 0$ ใด ๆ จงพิสูจน์ว่า $a \leq f(a) + \varepsilon$
\end{exercise}

\begin{exercise}
จงพิสูจน์ว่ามีจำนวนจริง $z \in [0, 1]$ ที่ $f(z) = z$
\end{exercise}

\section{One-to-one Correspondence}

สำหรับส่วนนี้ ให้ $\N$ แทนเซตของจำนวนเต็มไม่ลบ คือ $\{0, 1, 2, \dots\}$

\begin{exercise}
จงหาฟังก์ชันหนึ่งต่อหนึ่งทั่วถึงจาก $\N$ ไปสู่ $\Z$ และพิสูจน์ความเป็นหนึ่งต่อหนึ่งและทั่วถึงของฟังก์ชันนั้น
\end{exercise}

\begin{exercise}
จงหาฟังก์ชันหนึ่งต่อหนึ่งทั่วถึงจาก $\Z$ ไปสู่ $\Q$ และพิสูจน์ความเป็นหนึ่งต่อหนึ่งและทั่วถึงของฟังก์ชันนั้น
\end{exercise}

\begin{exercise}
จงหาฟังก์ชันหนึ่งต่อหนึ่งทั่วถึงจาก $\N$ ไปสู่ $\N \times \N$ และพิสูจน์ความเป็นหนึ่งต่อหนึ่งและทั่วถึงของฟังก์ชันนั้น
\end{exercise}

\begin{bonus}
จงพิสูจน์ว่าไม่สามารถหาฟังก์ชันทั่วถึงจาก $\N$ ไปสู่ $\R$ ได้
\end{bonus}