\chapter{Basic Dynamic Programming}

\section{Travelling Salesman Problem}

ปัญหา TSP เป็นปัญหาที่ถามว่า หากกำหนดกราฟไม่ระบุทิศทางแต่ถ่วงนำ้หนักมาให้กราฟหนึ่ง คือ $G = (V, E)$ ประกอบด้วย $N$ จุดยอด และ $M$ เส้นเชื่อม เซลล์แมนต้องการเดินทางไปให้ครบทุกจุดยอด จะเดินทางอย่างไรให้ผลรวมนำ้หนักบนการเดินทางนั้นมีค่าน้อยที่สุด

กล่าวคือ หากให้ $dist \colon V \times V \to \R$ แทนระยะทางที่สั้นที่สุดจากสองจุดบนกราฟ และให้ $\phi \colon V \to V$ เป็นฟังก์ชันหนึ่งต่อหนึ่งทั่วถึงแล้ว ปัญหา TSP ต้องการหาค่า
\[
\min_{\phi} \sum_{i=1}^{N-1} dist(\phi(i), \phi(i+1))
\]

เนื่องจาก $dist$ นั้นคำนวณมาก่อนได้ (มันคือปัญหา APSP: All-Pairs Shortest Path) ในเวลาเพียง $O(N^3)$ ส่วนใหญ่จึงให้ $M = \frac{N(N-1)}{2}$ และ $G$ เป็นกราฟบริบูรณ์ไปเลย แล้วทำการตอบปัญหา TSP ในรูปแบบ
\[
\min_{\phi} \sum_{i=1}^{N-1} D_{\phi(i), \phi(i+1)}
\]
แทน เมื่อ $D$ เป็นตารางสองมิติขนาด $N \times N$

\begin{exercise}
จงออกแบบอัลกอริทึมที่แก้ปัญหา TSP ได้ใน $O(2^N N^2)$
\end{exercise}

นอกจากนี้ยังมีปัญหา TSP ที่ปรับให้มีเงื่อนไขพิเศษ คือให้จุดแต่ละจุดอยู่บนระนาบสองมิติและใช้ระยะทางแบบยูคลิด (Euclidean Distance) แล้วบังคับว่าจะต้องเดินทางจากจุดซ้ายสุด ไปยังจุดขวาสุดก่อน แล้วค่อยย้อนกลับมา (ไม่สามารถเดินซ้ายสลับขวามั่ว ๆ ได้) เรียกปัญหานี้ว่า Bitonic TSP

\begin{exercise}
จงออกแบบอัลกอริทึมที่แก้ปัญหา Bitonic TSP ได้ใน $O(N^2)$
\end{exercise}

\begin{bonus}
มีอัลกอริทึมที่แก้ปัญหา Bitonic TSP ที่ดีกว่า $O(N^2)$ หรือไม่?
\end{bonus}

\section{Ternary Tree Traversal}
\label{ipst:o59_oct_c1_ternary}

ต้นไม้ตรีภาค คือต้นไม้ที่ปมแต่ละปมนั้นอาจจะมีลูกได้สามปม ได้แก่ ลูกซ้าย ลูกกลาง และ ลูกขวา กำหนดให้การหา preorder traversal และ inorder traversal ของปมในต้นไม้ต้นนี้เขียนเป็นรหัสเทียมได้ดังนี้ โดยให้ \texttt{left\_child(x)}, \texttt{center\_child(x)} และ \texttt{right\_child(x)} คือลูกซ้าย ลูกกลาง และ ลูกขวา ของปม \texttt{x} ตามลำดับ

\noindent\begin{minipage}{.48\textwidth}
\begin{lstlisting}[caption=Preorder]{Preorder}
preorder(node x) {
  if (x == NULL) return
  print(value(x))
  preorder(left_child(x))
  preorder(center_child(x))
  preorder(right_child(x))
}
\end{lstlisting}
\end{minipage}\hfill
\begin{minipage}{.48\textwidth}
\begin{lstlisting}[caption=Inorder]{Inorder}
inorder(node x) {
  if (x == NULL) return
  inorder(left_child(x))
  print(value(x))
  inorder(center_child(x))
  inorder(right_child(x))
}
\end{lstlisting}
\end{minipage}

เรามีต้นไม้ตรีภาคขนาด $n$ ปมอยู่ต้นหนึ่ง ซึ่งแต่ละปมนั้นกำกับด้วยจำนวนเต็มหนึ่งตัว ซึ่งมีค่าตั้งแต่ $0$ ถึง $n-1$  โดยที่แต่ละปมไม่ซำ้กันเลย จากข้อมูลผลการทำงานของ preorder traversal และ inorder traversal ที่ปมรากของต้นไม้ต้นนี้ จงคำนวณว่ามีต้นไม้กี่ต้น (รวมทั้งต้นไม้ของเรานี้ด้วย) ที่มีผลการทำงาน preorder และ inorder ตรงกับที่กำหนดให้

เนื่องจากคำตอบอาจมีค่าสูงมาก ให้ตอบเฉพาะจำนวนหารเอาเศษด้วย $10^9+7$

(โจทย์จากข้อ \texttt{o59\_oct\_c1\_ternary})

\begin{exercise}
จงพิสูจน์ว่า หากจำนวนเต็มตัวแรกของ preorder traversal มีค่าเท่ากับจำนวนเต็มตัวสุดท้ายของ inorder traversal แล้วรากของต้นไม้นี้จะไม่มีลูกตรงกลางและลูกทางขวา
\end{exercise}

\begin{exercise}
จงพิสูจน์ว่า หากจำนวนเต็มตัวแรกของ preorder traversal มีค่าเท่ากับจำนวนเต็มตัวแรกของ inorder traversal แล้วรากของต้นไม้นี้จะไม่มีลูกทางซ้าย
\end{exercise}

\begin{exercise}
จงออกแบบอัลกอริทึมในการแก้ปัญหาข้อนี้ภายในเวลา $O(N^4)$ และใช้ความจำ $O(N^3)$
\end{exercise}

\begin{bonus}
มีวิธีที่เร็วกว่า $O(N^4)$ หรือใช้ความจำที่น้อยกว่า $O(N^3)$ หรือไม่?
\end{bonus}

\section{Incantation}
\label{ipst:o60_oct_c2_incantation}

จอมเวทย์คนหนึ่งจะร่ายมนต์ มนต์ประกอบด้วยคำพูดติดกัน $n$ พยางค์ พยางค์แต่ละพยางค์เป็นไปได้สองแบบคือ \texttt{IN} และ \texttt{ORT} ตัวอย่างมนต์ที่ประกอบด้วยสามพยางค์คือ \texttt{IN ORT IN} หรือ \texttt{ORT ORT ORT} เป็นต้น ความแข็งแกร่งของมนต์นั้นขึ้นอยู่กับลำดับของพยางค์ต่าง ๆ ที่ปรากฎอยู่ในมนต์

จองเวทย์ศึกษาจนรู้ว่า ความแข็งแกร่งของมนต์นั้นสามารถคำนวณได้จากตารางความแข็งแกร่งของคำใดๆ ก็ตามที่มีความยาวเท่ากับ $k$ กล่าวคือ คำที่มีความยาว $k$ พยางค์ใด ๆ จะมีค่าความแข็งแกร่งอยู่ค่าหนึ่ง ความแข็งแกร่งของมนต์จะเท่ากับผลรวมของจำนวนครั้งที่คำความยาว $k$ นั้นปรากฎขึ้นคูณด้วยค่าความแข็งแกร่งของคำนั้นจากตารางที่กำหนดให้ สำหรับคำความยาว $k$ ทุกคำที่เป็นไปได้นั่นเอง

ตัวอย่างเช่น เมื่อ $n = 5$ และ $k = 2$ มนต์ \texttt{IN ORT IN ORT ORT} จะมีค่าความแข็งแกร่งเท่ากับ $(\text{ความแข็งแกร่งของ } \texttt{IN ORT} \cdot 2) + (\text{ความแข็งแกร่งของ } \texttt{ORT IN} \cdot 1) + (\text{ความแข็งแกร่งของ } \texttt{ORT ORT} \cdot 1)$ เป็นต้น

จากตารางความแข็งแกร่งของคำความยาว $k$ พยางค์ทั้งหมดที่กำหนดให้ จงหาว่ามนต์ที่แข็งแกร่งที่สุดที่เป็นไปได้นั้นมีค่าความแข็งแกร่งเท่าไร

(โจทย์จากข้อ \texttt{o60\_oct\_c2\_incantation})

\begin{exercise}
จงออกแบบอัลกอริทึมที่แก้ปัญหานี้ได้ใน $O(nk)$
\end{exercise}

\begin{exercise}
จงออกแบบอัลกอริทึมที่แก้ปัญหานี้ได้ใน $O(8^k \log n)$
\end{exercise}

\begin{bonus}
มีวิธีที่ดีกว่านี้อีกหรือไม่?
\end{bonus}