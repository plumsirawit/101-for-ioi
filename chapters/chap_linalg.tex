\chapter{Linear Algebra}

\renewcommand{\ker}{\rm{Ker}}
\newcommand{\im}{\rm{Im}}

\section{Polynomial Delta}

ก่อนจะเข้าสู่ปัญหานี้ เราจะพูดถึงสัญลักษณ์ต่าง ๆ ก่อน กำหนดให้ $\R[X]$ แทนปริภูมิเวกเตอร์ของพหุนามทั้งหมดที่มีสัมประสิทธิ์เป็นจำนวนจริง นอกจากนี้ สำหรับจำนวนเต็มไม่ลบ $n$ เราจะให้ $\R_n[X]$ แทนพหุนามใน $\R[X]$ ที่มีดีกรีไม่เกิน $n$ โดยที่ $\R_n[X]$ เป็นปริภูมิเวกเตอร์มิติ $n+1$ ซึ่งเป็นปริภูมิย่อยของ $\R[X]$ และฐานหลักมาตรฐานคือ $(1, X, \dots, X^n)$

หาก $k$ เป็นจำนวนเต็มบวก เรานิยาม $\Delta^k = \Delta \circ \Delta^{k-1}$ และ $\Delta^0$ คือฟังก์ชัน $P \mapsto P$ ซึ่งนิยามไว้สำหรับ $P \in \R[X]$ ใด ๆ

เป้าหมายของข้อนี้คือการทำการศึกษาการแปลงเชิงเส้น $\Delta \colon \R[X] \to \R[X]$ นิยามโดย
\[
\Delta(P) := P(X+1) - P(X)
\]
เมื่อ $P$ เป็นพหุนามใน $\R[X]$ ใด ๆ

\begin{exercise}
ให้ $(P_k)_{k \in \N}$ เป็นชุดของพหุนามจริงที่ $\deg(P_k) = k$ สำหรับทุก $k \in \N$ จงแสดงว่าสำหรับ $n \in \N$ ใด ๆ $(P_0, \dots, P_n)$ เป็นฐานหลักของ $\R_n[X]$
\end{exercise}

\begin{exercise}
จงแสดงว่า $\Delta$ เป็นการแปลงเชิงเส้นบน $\R[X]$
\end{exercise}

\begin{exercise}
นิยาม $\Delta_n$ โดย $\Delta_n(P) := \Delta(P)$ สำหรับทุก $P \in \R_n[X]$ จงแสดงว่า $\Delta_n$ เป็นการแปลงเชิงเส้นจาก $\R_n[X]$ สู่ $\R_n[X]$
\end{exercise}

\begin{exercise}
จงหาเมตริกซ์ของการแปลงเชิงเส้น $\Delta_2, \Delta_3, \Delta_4$ บนฐานหลักมาตรฐาน เมตริกซ์สำหรับการแปลงเชิงเส้น $\Delta_n$ จะเป็นอย่างไรในกรณีทั่วไป?
\end{exercise}

\begin{exercise}
จงหา $\ker(\Delta)$
\end{exercise}

\begin{exercise}
จงหา $\im(\Delta_n)$ และแสดงว่า $\Delta$ เป็นฟังก์ชันทั่วถึง
\end{exercise}

\section{Maximum subset XOR}

\begin{definition}
สำหรับจำนวนเต็มไม่ลบ $a$ และ $b$ ขนาด $m$ บิต ที่เขียนอยู่ในรูป
\[
a = \sum_{i=0}^{m-1} a_i 2^i, \qquad b = \sum_{i=0}^{m-1} b_i 2^i
\]
โดย $a_i, b_i \in \{0, 1\}$ สำหรับ $i \in \{0, 1, \dots, m-1\}$ ใด ๆ นิยาม 
\[
\xor(a, b) := \sum_{i=0}^{m-1}( (a_i + b_i) \mod 2) 2^i
\]
\end{definition}

\begin{definition}
สำหรับเซต $S$ ของจำนวนเต็มไม่ลบขนาด $m$ บิต โดยมีขนาดจำกัด เราจะนิยาม
\[
\xor_{x \in S} x := \xor(\min(S), \xor_{x \in S - \{\min(S)\}} x)
\]
หาก $|S| > 0$ และ $\xor\limits_{x \in \emptyset} x := 0$
\end{definition}

ให้ $m, N$ เป็นจำนวนเต็มที่มากกว่าหนึ่ง กำหนดเซต $S$ ให้ประกอบด้วยจำนวนเต็ม $N$ ตัวที่แตกต่างกัน โดยจำนวนเต็มแต่ละตัวมีค่าระหว่าง $0$ ถึง $2^m-1$ (มองว่าเป็นจำนวนเต็มไม่ลบ $m$ บิต)

เป้าหมายของข้อนี้คือการหาเซตย่อย $T \subseteq S$ ที่ $\xor\limits_{x \in T} x$ มีค่ามากที่สุด

\begin{exercise}
นิยามเซต $A = \{0, 1, \dots, 2^m-1\}$ จงพิสูจน์ว่าเซต $A$ ประกอบกับการดำเนินการบวกนิยามโดย $\xor \colon A \times A \to A$ และการคูณภายนอกจากฟีลด์ $\{0, 1\}$ นิยามตามการคูณกันของจำนวนเต็ม นั้นเป็นปริภูมิเวกเตอร์
\end{exercise}

\begin{exercise}
สำหรับ $S \subseteq A$ ที่กำหนดให้ในข้อนี้ จงแสดงว่า $S' := \spn(S)$ เป็นปริภูมิย่อยของ $A$ และ $S \subseteq S'$
\end{exercise}

\begin{exercise}
จงแสดงว่า
\[
\{\xor_{x \in T} T \colon T \subseteq S\} = \spn(B)
\]
เมื่อ $B$ เป็นฐานหลักของ $S'$ แบบใดก็ได้
\end{exercise}

\begin{exercise}
จงอธิบายอัลกอริทึมในการแก้ปัญหานี้ใน $O(m^2N)$ (คำใบ้: $|B| \in O(m)$)
\end{exercise}