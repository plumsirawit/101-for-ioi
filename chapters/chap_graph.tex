\chapter{Graph algorithms + graph theory}

\section{Black-White Graph Coloring}

ในส่วนนี้จะพิจารณาเฉพาะกราฟไม่ระบุทิศทางที่ไม่มี self-loop (กล่าวคือไม่มีจุดยอดใดมีเส้นเชื่อมกับตัวเอง) เราจะทำการศึกษาวิธีการระบายสีกราฟที่สอดคล้องกับเงื่อนไขต่อไปนี้
\begin{enumerate}[nosep]
    \item จุดยอดทุกจุดยอดจะมีสีกำหนดอยู่ เป็นอย่างใดอย่างหนึ่งระหว่างสีขาวและสีดำ
    \item สำหรับจุดยอด $u$ และ $v$ ที่อยู่ติดกัน (กล่าวคือ มีเส้นเชื่อม $\{u, v\} \in E$) สีของ $u$ กับ $v$ นั้นต่างกัน
\end{enumerate}

\begin{exercise}
จงพิสูจน์ว่าในกราฟเส้นตรง (กล่าวคือ $G = (V, E)$ เมื่อ $V = \{1, \dots, N\}$ และ $E = \{\{i, i+1\} \colon i \in \{1, \dots, N-1\}\}$ สำหรับจำนวนเต็มบวก $N$ ใด ๆ) จะมีวิธีการระบายสีกราฟตามเงื่อนไขข้างต้นอยู่ $2$ วิธี
\end{exercise}

\begin{exercise}
จงพิสูจน์ว่าหากกราฟไม่ใช่ Bipartite Graph แล้วจะไม่มีวิธีการระบายสีตามเงื่อนไขดังกล่าว (คำใบ้: พิสูจน์โดยแย้งสลับที่)
\end{exercise}

\begin{exercise}
จงพิสูจน์ว่าหากกราฟเป็น Bipartite Graph ที่เชื่อมต่อกัน (connected) แล้วจะมีวิธีการระบายสีตามเงื่อนไขอยู่ $2$ วิธี
\end{exercise}

\begin{exercise}
จงพิสูจน์ว่าหากกราฟ $G$ เกิดจากการนำกราฟ $G_1$ มารวม (union) กับกราฟ $G_2$ และกราฟ $G_1$ และ $G_2$ มีวิธีการระบายสีอยู่ $c_1$ แบบและ $c_2$ แบบตามลำดับ แล้ววิธีการระบายสีบนกราฟ $G$ จะมีอยู่ทั้งหมด $c_1 \cdot c_2$ แบบ
\end{exercise}

\begin{exercise}
จงพิสูจน์ข้อสรุป ว่าใน Bipartite Graph ที่ประกอบด้วย $C$ ส่วนเชื่อมต่อกัน (connected component) จะมีวิธีการระบายสีทั้งหมด $2^C$ แบบ
\end{exercise}

\section{DFS and Topological Sort}

ในส่วนนี้จะใช้ $N$ แทนจำนวนจุดยอดและ $M$ แทนจำนวนเส้นเชื่อม

นอกจาก Kahn's Algorithm ที่ใช้ในการหา Topological Ordering บน DAG ใน $O(N+M)$ ได้แล้ว เราจะพิจารณาอัลกอริทึมที่ค่อนข้างง่ายอีกอัลกอริทึมหนึ่ง ตังต่อไปนี้

\begin{enumerate}[nosep]
    \item สร้างสแตคเปล่า ชื่อ $S$
    \item สำหรับแต่ละจุดยอดในกราฟ ให้ทำการ DFS จากจุดยอดนั้น จนกว่าจะหมดกราฟโดยเมื่อทำ DFS เสร็จจะทำการใส่ของลงในสแตค
    \item ค่อย ๆ นำของออกจาก $S$ จากบนสุดของสแตคจนถึงล่างสุด ลำดับของจุดยอดที่นำออกมา จะสอดคล้องกับลำดับ Topological Order
\end{enumerate}

หรือพิจารณาโค้ดตัวอย่างดังต่อไปนี้

\begin{lstlisting}
std::vector<int> G[N+1]; // adjacency list
std::stack<int> S;
bool found[N+1]; // initially set to false
void dfs(int u){
    if(found[u]) return;
    for(int v : G[u]){
        dfs(v);
    }
    S.push(u);
}
\end{lstlisting}

และเมื่อได้กราฟมาแล้วจะทำการรันดังต่อไปนี้

\begin{lstlisting}
for(int u = 1; u <= N; u++){
    dfs(u);
}
while(!S.empty()){
    std::cout << S.top() << std::endl;
    S.pop();
}
\end{lstlisting}

\begin{exercise}
จงพิสูจน์ว่าอัลกอริทึมดังกล่าวให้ผลลัพธ์ออกมาเป็น Topological Order และใช้เวลาใน $O(M+N)$
\end{exercise}

\begin{bonus}
มีอัลกอริทึมที่สามารถหา Topological Order ที่ Lexicographical Order น้อยสุดที่เป็นไปได้สำหรับทุก Topological Order ที่เป็นไปได้หรือไม่ในเวลา $O(M+N)$
\end{bonus}

\section{Sun Graphs}

\begin{exercise}
สำหรับกราฟที่เชื่อมต่อกันที่เป็น simple graph (ไม่มีลูปวนซำ้ตัวเอง และไม่มีเส้นเชื่อมขนาน) ที่มีจำนวนจุดยอดเท่ากับจำนวนเส้นเชื่อม จงพิสูจน์ว่ากราฟนี้จะมีวัฏจักร (cycle) อยู่ $1$ อันพอดี และเมื่อลบวัฏจักรนี้ทิ้งออกไปจะเหลือเป็นป่า (forest) (หมายถึง กราฟที่ประกอบด้วยต้นไม้หลายต้น หรือต้นเดียว หรือไม่มีต้นไม้เลยก็เป็นได้)

ในส่วนนี้จะขอเรียกกราฟประเภทนี้ว่ากราฟดวงอาทิตย์ (ไม่ใช่ชื่อเรียกทางการ; ไม่สามารถใช้นอกขอบเขตนี้ได้)
\end{exercise}

\begin{exercise}
จงพิสูจน์ว่าสำหรับกราฟที่เป็น simple graph แต่อาจไม่เชื่อมต่อกันนั้น เราสามารถแบ่งกราฟนี้เป็นผลรวม (union) ของกราฟดวงอาทิตย์หลาย ๆ กราฟได้
\end{exercise}

\section{Smallest Fibonacci-covered-difference Set}
\label{imo:2020slc4}

นิยามจำนวนฟิโบนัชชี $F_0, F_1, \dots$ จาก $F_0 = 0$, $F_1 = 1$ และ $F_{n+1} = F_{n} + F_{n-1}$ สำหรับทุกจำนวนเต็ม $n \geq 1$ ให้จำนวนเต็ม $n \geq 2$ จงหาขนาดที่เล็กที่สุดที่เป็นไปได้ของเซต $S$ ที่สำหรับทุก $k = 2, 3, \dots, n$ จะมีบาง $x, y \in S$ ที่ $x - y = F_k$

(โจทย์จาก IMO 2020 Shortlist ข้อ C4.)

สำหรับเซต $S \subseteq \Z$ ใด ๆ ที่สอดคล้องกับเงื่อนไข เราจะทำการสร้างกราฟดังนี้ ให้ $G$ เป็นกราฟไม่ระบุทิศทางถ่วงนำ้หนัก โดยมีเส้นเชื่อมเป็นเซตเดียวกับ $S$ และสำหรับทุกจำนวนเต็ม $1 \leq k \leq \lceil \frac{n}{2} \rceil$ เลือก $x, y \in S$  ที่ $x - y = F_{2k+1}$ แล้วเติมเส้นเชื่อม $\{x, y\}$ ลงไปในกราฟ

\begin{exercise}
จงพิสูจน์ว่าสำหรับแต่ละจำนวนเต็ม $1 \leq k \leq \lceil \frac{n}{2} \rceil$ จะสามารถเลือก $x, y \in S$ ที่ $x - y = F_{2k+1}$ ได้เสมอ
\end{exercise}

\begin{exercise}
จงพิสูจน์ว่ากราฟดังกล่าวไม่มีวัฏจักร (cycle)
\end{exercise}

\begin{exercise}
จงพิสูจน์ว่าสำหรับทุกเซต $S \subseteq \Z$ ที่สอดคล้องกับเงื่อนไข จะได้ว่า $|S| \geq \lceil \frac{n}{2} \rceil + 1$
\end{exercise}

\begin{exercise}
จงพิสูจน์ว่ามีเซต $S$ ที่สอดคล้องกับเงื่อนไขที่ $|S| = \lceil \frac{n}{2} \rceil + 1$ และสรุปผล
\end{exercise}