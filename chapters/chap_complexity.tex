\chapter{Complexity Analysis, Sorting \& Searching}

\section{A Big Factorial}

\begin{exercise}
ให้ $g \colon \N \to \R$ นิยามโดย $g(n) = \log(n!)$ สำหรับ $n \in \N$ ใด ๆ จงแสดงว่า $g(n) \in \Theta(n \log n)$
\end{exercise}

\section{An Impossible Problem}
\label{twitter:willmcgugan1}
กำหนดให้มีอาเรย์ของจำนวนเต็มที่อาจเป็น 0, 1 หรือ 2 ก็ได้ ขนาด $n$ เช่น $[0, 1, 2, 0, 1]$

จงออกแบบอัลกอริทึมที่จะแบ่งอาเรย์นี้เป็นสองส่วนโดยการตัดออกเป็นส่วนที่ติดกันทางซ้าย เรียกว่า $A$ และส่วนที่ติดกันทางขวา เรียกว่า $B$ โดยมีเงื่อนไขว่า ผลรวมของจำนวนใน $A$ จะต้องมีค่าเท่ากับ $X$ และ $B$ จะต้องไม่เริ่มต้นด้วย $0$ (เพื่อความง่ายขอรับประกันว่ามีคำตอบแน่นอน)

โดยอัลกอริทึมจะต้องมีประสิทธิภาพ กล่าวคือ ทำงานได้เร็วกว่า $O(n)$

\begin{exercise}
จงพิสูจน์ว่ามันทำไม่ได้ กล่าวคือ สำหรับอัลกอริทึมใด ๆ ก็ตามที่สามารถแก้ปัญหาข้างต้นได้ อัลกอริทึมนั้นจะต้องใช้เวลา $\Omega(n)$ (Hint: พิจารณาเฉพาะเงื่อนไขของ $B$)
\end{exercise}

\begin{exercise}
สมมติว่าตัดเงื่อนไขของ $B$ ออก เหลือเพียงเงื่อนไขว่าผลรวมของจำนวนใน $A$ จะต้องมีค่าเท่ากับ $X$ จงแสดงว่าอัลกอริทึมที่แก้ปัญหานี้ได้จะต้องใช้เวลา $\Omega(n)$
\end{exercise}

\noindent \textbf{หมายเหตุ.} โจทย์ดัดแปลงมาจาก \\ \url{https://twitter.com/willmcgugan/status/1459233690052141063?s=20}

\section{Sieve of Eratosthenes}

สำหรับจำนวนเต็มบวก $N$ พิจารณาอัลกอริทึมที่ตรวจสอบว่าจำนวนใดบ้างเป็นจำนวนเฉพาะตั้งแต่ $1$ ถึง $N$ ดังนี้

\begin{lstlisting}
bool is_prime[N+1];
is_prime[1] = false;
for(int i = 2; i <= N; i++){
    is_prime[i] = true;
}
for(int gap = 2; gap <= N; gap++){
    for(int i = 2*gap; i <= N; i += gap){
        is_prime[i] = false;
    }
}
\end{lstlisting}

\begin{exercise}
สังเกตว่า การวนซำ้ชั้นในจะทำเพียง $O\left(\frac{N}{\texttt{gap}}\right)$ รอบ ดังนั้นเวลารวมของอัลกอริทึมนี้จึงเป็น
\[
O\left(\sum_{\texttt{gap}=2}^N \frac{N}{\texttt{gap}}\right)
\]
จงพิสูจน์ว่า เวลารวมของอัลกอริทึมนี้ อยู่ใน $O(N \log N)$
\end{exercise}

\begin{bonus}
การดัดแปลงอัลกอริทึมข้างต้น ให้สนใจเฉพาะกรณีที่ \texttt{gap} เป็นจำนวนเฉพาะ สามารถทำให้เวลาที่อัลกอริทึมนี้ใช้ดีขึ้นไปอีก จงหาเวลาที่ใช้และพิสูจน์
\end{bonus}